% !Mode:: "TeX:UTF-8"
% 请务必使用 XeLaTeX 编译本文,如目录不显示,请二次编译
\documentclass{GZHUMaster}

%------------------------ 参考文献 ------------------------
\bibliographystyle{abbrv}        %参考文献样式,plain、unsrt、alpha、abbrv等等

%------------------------ 全篇开始 ------------------------
\begin{document}
\setmainfont{Times New Roman}  %根据广州大学研究生工作手册,学位论文的英文及字母书写应使用Times New Roman

%------------------------ 变量赋值 ------------------------
\SchoolNumber{11078}           %11078是广州大学的编号
\title{广州大学硕士毕业论文~\LaTeX~模板}                       %中文题目
\Etitle{A \LaTeX~Thesis Template for Guangzhou University} %英文题目
\author{作者}                   %作者姓名
\StudentNumber{1234567890}     %学号
\Supervisor{}                  %指导教师姓名
\Major{专业名称}                %专业名
\Specialty{方向名称}            %研究方向
\date{2077.1.1}               %日期只写年月
\Chairman{}                    %答辩委员会主席
\Member{}                      %答辩委员会成员

%------------------------ 封面书签 ------------------------
\pdfbookmark[0]{封面}{title}         %封面页加到pdf书签
\maketitle

%------------------------ 前置设置 ------------------------
\frontmatter
\pagenumbering{Roman}               %正文之前的页码用大写罗马字母编号
\cleardoublepage
\newpage
\pagestyle{fancy}
\fancyfancy

%------------------------ 摘要添加 ------------------------
% !Mode:: "TeX:UTF-8"
%此部分需要自己填写的内容:中文摘要及关键词、英文摘要及关键词
\renewcommand{\baselinestretch}{1.5}  %正文行距1.5倍

%------------------------ 中文摘要 ------------------------
\begin{cnabstract}
本文主要介绍了与本~\LaTeX~模板相关的使用事宜,
指明了编译方法,强调了公式图片表格排版的一些细节问题,还有各种其他潜在的问题。
\par
本模板基于【武汉大学黄正华】同学的开源模板修改而来,修改过程中参考了广州大学几位硕士博士的学位论文终稿
样式和广州大学学位与研究生教育工作手册中的学位论文规范。
\par
本模板全程采用UTF-8编码制作而成,理论上可以在任何系统和软件的TeX环境下使用XeLaTex成功编译。
%----------------------- 中文关键词 -----------------------
\\\\
\cnkeywords{毕业论文;\LaTeX;模板;XeLaTeX;编码}
\end{cnabstract}

%------------------------ 英文摘要 ------------------------
\begin{enabstract}
This thesis is mainly about the use of this \LaTeX template. It indicates the compiling method, emphasizing 
some detailed and potential problems about the layout design of formulas, images and tables.
\par
This template is based on an open source template from Zhenghua Huang of WHU, with referring to some 
thesis of GZHU and the criterion from GZHU workbook.  
\par
This template is encoded with UTF-8, which means you can successfully compile it under any 
TeX environment on any platform with XeLaTeX.
%----------------------- 英文关键词 -----------------------
\\\\
\enkeywords{Thesis, \LaTeX, Template, XeLaTeX, Encoding}
\end{enabstract}      %加入中英文摘要
%%%\renewcommand{\chaptermark}[1]{\markboth{#1}{}}

%------------------------ 目录添加 ------------------------
\pdfbookmark[0]{目录}{toc}
{\zihao{-4}\tableofcontents}

%------------------------ 正文设置 ------------------------
\mainmatter %以下是正文
\renewcommand{\baselinestretch}{1.5}  %正文行距1.5倍
\zihao{-4}                            %正文字号小四号
\renewcommand{\thefigure}{\thechapter-\arabic{figure}} %将图标题序号改写为 1-1 的形式,若希望 1.1 ,请注释掉本行
\renewcommand{\thetable}{\thechapter-\arabic{table}}   %将表标题序号改写为 1-1 的形式,若希望 1.1 ,请注释掉本行
%%%\newcommand{\zhchapter}{\zhnumber{\thechapter}}        %若页眉无法呈现 第一章 的形式,请取消注释由 %%% 注释的行,分别位于【摘要添加】、【正文设置】和【参考文献】
%%%\renewcommand{\chaptermark}[1]{\markboth{第 \zhchapter 章 \quad #1}{}}

%%%%%%%%%%%%%%%%%%%%%%%%%%第一章%%%%%%%%%%%%%%%%%%%%%%%%%%
\chapter{使用方法}
对于不熟悉~\LaTeX 的同学,建议在全面完成毕业论文初稿后使用本模板。\LaTeX 上手需要一定时间,调整图表公式格式和位置是一个辛苦活,但最终结果是美好的!
\section{环境搭建}
\subsection{软件下载地址}
基于本模板的测试情况,强烈推荐采用MiKTeX配合TeXstudio的使用方式,这种方式最稳定也最高效。\par
MiKTeX下载地址:\par
\url{https://miktex.org/download}\par
TeXstudio下载地址:\par
\url{https://texstudio.sourceforge.net}\par
请务必根据自己的系统情况选择合适的版本进行下载!!\par
\subsection{安装及配置步骤}
下载完成后,请按以下步骤安装配置所需软件:\par
(1)安装MiKTeX,按默认选项安装即可。\par
(2)打开MiKTeX Console,检查是否安装成功。\par
(3)安装TeXstudio,按默认选项安装即可。\par
(4)打开TeXstudio,进入设置,选择“构建”部分,将右侧的“默认编译器”由PdfLaTeX改为XeLaTeX。\par
至此,TeX软件的安装配置工作已经完成,但编译文档所需的各种宏包还未安装,详情请参阅后续步骤。
\section{具体使用步骤}
第一步:进入includefile文件夹,打开midmatter.tex填写中文摘要和英文摘要,然后打开backmatter.tex填写致谢。\par
第二步:打开主文档Template.tex,填写个人信息并书写正文。\par
第三步:使用XeLaTeX编译本文,记得编译两次,否则生成的文档中没有目录!\par
注意事项:若使用本文推荐的软件环境,在第一次编译时会提示缺少宏包,请根据指引慢慢安装这些宏包,安装完毕后再次编译就不会出现这种问题了。
\section{其他问题}
本模板默认新章节内容从右手边开始,因此可能造成许多空白页,若希望在正文中消除这些空白页,请在GZHUMaster.cls的开头找到:\par
\verb|\LoadClass[a4paper,twoside]{ctexbook}|\par
并改为:\par
\verb|\LoadClass[a4paper,twoside,openany]{ctexbook}|\par
除此之外,模板中还有许多可以自行调节的地方,详细内容请参考tex和cls文件中的注释。\par
若存在其他潜在的编译问题,请参考本人开发及测试环境,如表 \ref{tab1} 所示,在这些环境下本文均可正常编译。请注意,macOS、Windows和Linux的字体库各不相同,同一个文档在不同系统下生成的字体外观存在细微差异,这是正常的。\par
\begin{table}[h]
  \centering
  \caption{开发及测试环境}
  \begin{tabular}{ccc}
    \hline
    & 开发环境 & 测试环境 \\
    \hline
    OS & macOS Sequoia & Windows 10 \\
    TeX版本 & MacTeX-2024 & MiKTeX-24.1 \\
    开发环境 & Visual Studio Code 1.96.2 & TeXstudio 4.8.5 \\
    编译器 & XeLaTeX & XeLaTeX \\
    \hline
  \end{tabular}
  \label{tab1}
\end{table}
除此之外,本文档也在Overleaf在线LaTeX编辑器中完成编译,若不方便下载上述软件,也可以在Overleaf中进行编辑。
%%%%%%%%%%%%%%%%%%%%%%%%%%第二章%%%%%%%%%%%%%%%%%%%%%%%%%%
\chapter{常用功能介绍}
\section{正文书写}
正文的具体书写方式请参考本文~.tex~文件中的代码,以下仅介绍部分常用命令:\par
章标题:\verb|\chapter{}|\par
节标题:\verb|\section{}|、\verb|\subsection{}|、\verb|\subsubsection{}|\par
换行(段落不变):\verb|\\|\par
换行(另起一段):\verb|\par|\par
加粗(\textbf{Bold}):\verb|\textbf{}|\par
斜体(\textit{Italic}):\verb|\textit{}|\par
宋体({\songti 宋体}):\verb|\songti|\par
黑体({\heiti 黑体}):\verb|\heiti|,请注意,对宋体使用加粗命令即可变为黑体\par
仿宋({\fangsong 仿宋}):\verb|\fangsong|\par
楷书({\kaishu 楷书}):\verb|\kaishu|,请注意,对宋体使用斜体命令即可变为楷书\par
字号:\verb|\zihao{}|,大括号内填~-4为小四号,填4为四号字,以此类推
\section{插入表格}
\subsection{一般三线表}
普通的三线表的绘制方式如表 \ref{tab2} 所示,仅作为表格编写示范,不建议在毕业论文中使用。
\begin{table}[h]
  \centering
  \caption{一般三线表}
  \begin{tabular}{ccc}
    \hline
    列1 & 列2 & 列3 \\
    \hline
    1 & 2 & 3 \\
    4 & 5 & 6 \\
    7 & 8 & 9 \\
    \hline
  \end{tabular}
  \label{tab2}
\end{table}
\subsection{存在合并列和双上下框线的表格}
存在单元格合并情况的表格处理起来复杂一些,如表 \ref{tab3} 所示,此表为顶部底部双线表,中间的线用于展示表格关系,请按需使用。
\begin{table}[h]
  \centering
  \caption{单元格合并的表格}
  \begin{tabular}{c|c|c}
    \hline\hline
    \multicolumn{2}{c|}{合并列} & 独立列 \\
    \hline
    & 1 & 2 \\\cline{2-3}
    合并行 & 3 & 4 \\\cline{2-3}
    & 5 & 6 \\
    \hline\hline
  \end{tabular}
  \label{tab3}
\end{table}

\subsection{标准毕业论文表格}
标准的毕业论文表格建议符合以下要求:\par
(1)表格与页面文字同宽。\par
(2)上下框线加粗。\par
(3)内容居中。\par
此处以tabularx包的用法为例,使用时请以表 \ref{tab4} 的方式设计。在进行列设定时,请注意一个\verb|>{\centering\arraybackslash}X|代表一列(自动设置列宽并居中的意思)。
\begin{table}[h]
  \centering
  \caption{标准毕业论文三线表}
  \begin{tabularx}{\textwidth}{>{\centering\arraybackslash}X>{\centering\arraybackslash}X>{\centering\arraybackslash}X}
    \toprule
    列1 & 列2 & 列3 \\
    \midrule
    1 & 2 & 3 \\
    4 & 5 & 6 \\
    7 & 8 & 9 \\
    \bottomrule
  \end{tabularx}
  \label{tab4}
\end{table}
\subsection{标准跨页表格}
如需引入符合标准的跨页表格,可按表 \ref{tab5} 的形式设计。\par

\begingroup\zihao{5}
\begin{longtable}{>{\centering\arraybackslash}p{0.307\textwidth}>{\centering\arraybackslash}p{0.307\textwidth}>{\centering\arraybackslash}p{0.307\textwidth}}
  \caption{跨页表格}
  \label{tab5}\\
  \toprule
  列1 & 列2 & 列3 \\
  \midrule
  \endfirsthead
  \toprule
  列1 & 列2 & 列3 \\
  \midrule
  \endhead
  \bottomrule
  \endfoot
  \bottomrule
  \endlastfoot
  1 & 2 & 3 \\
  4 & 5 & 6 \\
  7 & 8 & 9 \\
  10 & 11 & 12 \\
  13 & 14 & 15 \\
  16 & 17 & 18 \\
\end{longtable}
\endgroup

注意到此处使用 \verb|\begingroup\zihao{5}| 和 \verb|\endgroup| 来修改表格字体大小。\par
出于某些奇怪的原因,longtable的列宽参数并不能设置为相加为1的textwidth,根据测试,与页面文字同宽的参数为0.921,也就是需要保证每一列的宽度加起来为0.921。表 \ref{tab5} 中的每列参数均为0.307。
\subsection{进阶标准毕业论文表格}
一些表格可能存在行合并的情况,或者需要手动调整列宽,因此并不能简单设置为默认的三线表,具体情况如表 \ref{tab6} 所示。\par
\begin{table}[h]
  \centering
  \caption{进阶标准毕业论文表格}
  \begin{tabularx}{\textwidth}{>{\centering\arraybackslash\hsize=0.4\hsize}X>{\centering\arraybackslash\hsize=1.2\hsize}X>{\centering\arraybackslash\hsize=1.2\hsize}X>{\centering\arraybackslash\hsize=1.2\hsize}X}
    \toprule
    列1 & 列2 & 列3 & 列4\\
    \midrule
    A & 1 & 2 & 3\\\cmidrule{2-4}
    & 4 & 5 & 6\\
    B & 7 & 8 & 9 \\
    & 10 & 11 & 12\\\cmidrule{2-4}
    \raisebox{-0.9em}[0pt][0pt]{C} & 13 & 14 & 15\\
    & 16 & 17 & 18\\
    \midrule
    X & 1 & 2 & 3\\
    \bottomrule
  \end{tabularx}
  \label{tab6}
\end{table}
在此表中,第一列列宽为0.4倍,其他列为1.2倍,总共加起来仍为4。参数hsize可以自动按比例调整列宽,并不需要相加为列数本身。hsize默认值为1,即不填。
\section{插入图片}
插入图片的效果如图 \ref{img1} 所示。\par
\begin{figure}[h]
  \centering
  \includegraphics[width=\textwidth]{image1.jpeg}
  \caption{示例图片}
  \label{img1}
\end{figure}
这张图片的宽度与页面宽度一致。\par
再插入第二张图,自定义其宽度,如图 \ref{img2} 所示。\par
\begin{figure}[h]
  \centering
  \includegraphics[width=40em]{image2.jpeg}
  \caption{示例图片(宽度为40em)}
  \label{img2}
\end{figure}
尽管\verb|\begin{figure}[]|中括号中的参数填写为h,但因为版面空间不够,图片会显示在下一页,需要通过进一步调整图片宽度解决这些问题。参数h代表将图表放置于此,t代表放置顶部,b代表放置底部,p代表放置于下一页,可混合使用。
\section{插入公式}
通常情况下,想实现$x$这样的字体,只需要输入:\verb|$x$|即可。\par
对于$x_1$这样的下标,只需要输入:\verb|$x_1$|即可。\par
对于希腊字母$\beta$,只需要输入:\verb|$\beta$|即可。\par
对于公式的写法,可以参考式\ref{equ1}的写法。
\begin{equation}
  y=\alpha x^2. \tag*{(2-1)}
  \label{equ1}
\end{equation}\par
对于包含求和的公式,可以参考式\ref{equ2}的写法。
\begin{equation}
  y=\sum_{i=1}^{n}x_i. \tag*{(2-2)}
  \label{equ2}
\end{equation}\par
对于不需要参考的公式,无需对其进行编号:
\[y=\alpha x + \beta,\]
其中$\alpha$为斜率,$\beta$为截距。\par
请注意,规范书写的公式结尾应具备英文标点符号,作为段落中的一部分。如上所述,若参数介绍紧跟列出的公式,请顶格书写“其中\dots”等描述。\par
分数的书写方式像这样$\frac{4}{9}$,用法为\verb|$\frac{4}{9}$|。\par
其他数学符号写法请自行查阅资料。
\section{引用标记}
对于论文的引用,可以参考文献\cite{r1},也可以参考文献\upcite{r2}。\par
具体写法为\verb|\cite{r1}|和\verb|\upcite{r2}|。\par
另外可参考文献\upcite{r1,r2,r3}。\par
写法为\verb|\upcite{r1,r2,r3}|。
\section{插入脚注}
一些文字可能需要额外的解释\footnote{即脚注,就像这样。},使用脚注可以补充这些内容。也可手动控制序号\footnote[16]{就像这样!}。
\section{插入算法}
算法模块的排版参考广州大学魏伟明同学的开源模板,特此致谢。\par
以下是算法 \ref{alg1} 的伪代码:\par
\begin{algorithm}[h]
  \caption{算法名称}
  \begin{algorithmic}[1] %每行显示行号,删除则不显示
      \Require 这是输入
      \Ensure 这是输出
      \For{$j=0,\cdots, d$}
          \State {Test} \Comment{这是注释}
      \EndFor
      \If{$w>1$}
        \State {Test}
      \EndIf
      \While{$a>1$}
        \State{Test}
      \EndWhile
      \State{$\textbf{return }$ 0}
  \end{algorithmic}
  \label{alg1}
\end{algorithm}
可根据喜好自行改变字体与缩进方式。

%%%%%%%%%%%%%%%%%%%%%%%%%%第三章%%%%%%%%%%%%%%%%%%%%%%%%%%
\chapter{更新方式}
本章仅供之前已使用过本模板的情况下,初次使用无需阅读。
\section{检查更新}
本模板的更新会不定期发布于GitHub。\par
下载地址:\verb|https://github.com/swylll/GZHU_LaTeX_Master_Template|
\section{简易更新}
直接将新版本中的GZHUMaster.cls替换掉老版本中的相同文件即可。如需使用更新的功能请阅读Tutorial.pdf,具体代码仍需参考新版本的Template.tex文件。\par
注意:由于部分更新内容是包含在Template.tex主文件中的,因此不建议直接将GZHUMaster.cls文件替换为新版。若使用者已在cls文件或tex文件中加入自己所需的包或修改格式,请自行定位并移植这些修改部分。
\section{完整更新}
(1)同时打开新老版本文件夹,将老版本文件夹中的figures文件夹复制替换掉新版本的同名文件夹。\par
(2)同时打开新老版本tex主文件,将老版本从“正文设置”后到”参考文献“前的部分复制替换到新的tex文件模板中。\par
(3)对于参考文献,请仅替换\par
\verb|\begin{thebibliography}{0}\zihao{-4}|\par
到\par
\verb|\end{thebibliography}|\par
中的部分。

%------------------------ 参考文献 ------------------------
\cleardoublepage\phantomsection
%%%\renewcommand{\chaptermark}[1]{\markboth{#1}{}}
\makeatletter
\renewcommand{\@biblabel}[1]{\makebox[2em][l]{[#1]}}
\makeatother
\addcontentsline{toc}{chapter}{参考文献}
\begin{thebibliography}{0}\zihao{-4}
  \bibitem{r1} 作者,文章题目[类型],期刊名,年份(期数):起止页码
  \bibitem{r2} Authors,Title[Category],Name of the Journal,Year(Issue):Page
  \bibitem{r3} 作者1,作者2,文章题目[类型],期刊名,年份(期数):起止页码
\end{thebibliography}

%------------------------ 附录内容 ------------------------
\appendix
\chapter{附录使用方式}
以下为附录的编辑方式。
\section{附录编辑方式}
在\verb|\appendix|后的\verb|\chapter{}|以及\verb|\section{}|均以A、B、C的形式排布,不影响正文章节。


\backmatter
% !Mode:: "TeX:UTF-8"
%------------------------ 致谢部分 ------------------------
\acknowledgement
感谢所有与本模板相关的老师同学!\par
版本日志:\par
1.0.0\par
广大应统毕业论文模板的第一个公测版本,基本满足毕业论文的书写需求。\par
1.0.1\par
修复参考文献序号对齐问题。\par
1.1.0\par
增加多种表格设计形式,满足多种表格设计需求。\par
1.1.1\par
更新标准毕业论文表格设计。\par
1.1.2\par
微调参考文献格式。\par
1.2.0\par
加入有关更新方式的第三章,方便之前使用过本模板的同学更新到新版本。\par
1.3.0\par
加入附录功能,修改关键词格式。\par
1.4.0\par
修改目录格式,加入脚注功能。\par
1.4.1\par
增设参考文献连续标记,微调公式书写格式。\par
1.5.0\par
修改扉页格式为最新版。
\newpage
\cleardoublepage %致谢
\cleardoublepage
\end{document}